% \iffalse
\let\negmedspace\undefined
\let\negthickspace\undefined
\documentclass[beamer]{IEEEtran}
\usepackage{cite}
\usepackage{amsmath,amssymb,amsfonts,amsthm}
\usepackage{algorithmic}
\usepackage{graphicx}
\usepackage{textcomp}
\usepackage{xcolor}
\usepackage{txfonts}
\usepackage{listings}
\usepackage{enumitem}
\usepackage{mathtools}
\usepackage{gensymb}
\usepackage{comment}
\usepackage[breaklinks=true]{hyperref}
\usepackage{tkz-euclide} 
\usepackage{listings}
\usepackage{gvv}                                    
\usepackage{tikz}
\usepackage{pgfplots}
\def\inputGnumericTable{}                                 
\usepackage[latin1]{inputenc}                                
\usepackage{color}                                            
\usepackage{array}                                            
\usepackage{longtable}                                       
\usepackage{calc}                                             
\usepackage{multirow}                                         
\usepackage{hhline}                                           
\usepackage{ifthen}                                           
\usepackage{lscape}
\usepackage[export]{adjustbox}

\newtheorem{theorem}{Theorem}[section]
\newtheorem{problem}{Problem}
\newtheorem{proposition}{Proposition}[section]
\newtheorem{lemma}{Lemma}[section]
\newtheorem{corollary}[theorem]{Corollary}
\newtheorem{example}{Example}[section]
\newtheorem{definition}[problem]{Definition}
\newcommand{\BEQA}{\begin{eqnarray}}
\newcommand{\EEQA}{\end{eqnarray}}
\newcommand{\define}{\stackrel{\triangle}{=}}
\theoremstyle{remark}
\newtheorem{rem}{Remark}
\begin{document}
% Define custom function M(f)
\pgfmathdeclarefunction{Mf}{1}{%
  \pgfmathparse{sin(deg(#1))}%
}

\parindent 0px
\bibliographystyle{IEEEtran}

\title{GATE - EC 27}
\author{EE23BTECH11215 - Penmetsa Srikar Varma$^{}$% <-this % stops a space
}
\maketitle
\newpage
\bigskip

\renewcommand{\thefigure}{\theenumi}
\renewcommand{\thetable}{\theenumi}
\section*{Question}
Q27) Let m\brak{\text{t}} be a strictly band-limited signal with bandwidth B and energy E. Assuming $\omega_0$ = 10B, the energy in the signal $\text{m}\brak{\text{t}}\text{cos}\brak{\omega_0\text{t}}$\\[1ex]
\brak{\text{A}}\ $\frac{\text{E}}{4}$\\[1ex]
\brak{\text{B}}\ $\frac{\text{E}}{2}$\\[1ex]
\brak{\text{C}}\ \text{E}\\[1ex]
\brak{\text{D}}\ 2\text{E} \qquad\qquad\qquad\quad\qquad\qquad\qquad\qquad\brak{\text{GATE EC 2023}}
\section*{Solution}
\begin{table}[h]
    \centering
    \begin{tabular}{|c|c|c} 
    \hline
        Variables & Conditions \\
    \hline
        M\brak{\text{f}} & Fourier transform of m\brak{\text{t}}\\
    \hline
         \text{y}\brak{\text{t}} & \text{y}\brak{\text{t}}=$\text{m}\brak{\text{t}}\text{cos}\brak{2\pi\text{f}_0\text{t}}$\\
    \hline
        Y\brak{\text{f}} & Fourier transform of y\brak{\text{t}}\\
    \hline
    \end{tabular}

    \label{tab:my_label}
\end{table}
\begin{center}
    Table of Parameters\\
\end{center}
Let us assume for a case of M\brak{\text{f}},\\[1ex]
\begin{tikzpicture}
\begin{axis}[
    xlabel={f},
    ylabel={M\brak{\text{f}}},
    xmin=-1.5, xmax=1.5,
    ymin=0, ymax=1.2,
    axis lines=middle,
    minor tick num=1,
    width=10cm,
    height=6cm,
    xtick={-1,0,1},
    xticklabels={$-\text{B}$, 0, $\text{B}$},
    extra x ticks={-2,2},
    extra x tick labels={$-2B$, $2B$},
    extra x tick style={grid=none}
]
\addplot[blue, thick, domain=-1:1, samples=100] {1};
\draw[dotted] (axis cs:1,0) -- (axis cs:1,1);
\draw[dotted] (axis cs:-1,0) -- (axis cs:-1,1);
\end{axis}
\end{tikzpicture}\\[1ex]
Energy \brak{\text{E}} of the signal M\brak{\text{f}} is given by,
\begin{align}
\label{1}
    \text{E}&=\frac{1}{2\pi}\int_{-\text{B}}^{\text{B}}\abs{\text{M}\brak{\text{f}}}^2 \text{df}=\frac{\text{B}}{\pi}
\end{align}
Fourier transform of y\brak{\text{t}} is given by,
\begin{align}
    \text{Y}\brak{\text{f}}=\text{M}\brak{\text{f}}*\frac{1}{2}\brak{\delta\brak{\text{f}+\text{f}_0}+\delta\brak{\text{f}-\text{f}_0}}
\end{align}
\begin{align}
    \text{Y}\brak{\text{f}}=\frac{1}{2}\brak{\text{M}\brak{\text{f}+\text{f}_0}+\text{M}\brak{\text{f}-\text{f}_0}}
\end{align}
    
\begin{tikzpicture}[x=0.42cm, y=3cm] % Adjust the scale as needed
    % Axis
    \draw[->] (-12,0) -- (12,0) node[right] {f};
    \draw[->] (0,0) -- (0,1.5) node[above] {Y\brak{\text{f}}};
    
    % Horizontal lines
    \draw[line width=1.5pt] (-11,0.5) -- (-9,0.5);
    \draw[line width=1.5pt] (9,0.5) -- (11,0.5);
    
    % Labels
    \node[below] at (-11,0) {$-11\text{B}$};
    \node[below] at (-9,0) {$-9\text{B}$};
    \node[below] at (9,0) {$9\text{B}$};
    \node[below] at (11,0) {$11\text{B}$};
    
    % Set y-axis limits
    \draw[dashed] (-12,0.5) -- (12,0.5) node[right] {$\frac{1}{2}$};
    
    % Vertical dotted lines
    \draw[dotted] (-11,0) -- (-11,0.5);
    \draw[dotted] (-9,0) -- (-9,0.5);
    \draw[dotted] (9,0) -- (9,0.5);
    \draw[dotted] (11,0) -- (11,0.5);
\end{tikzpicture}\\[1ex]

Energy $\brak{\text{E}_1}$ of the signal Y\brak{\text{f}} is given by,
\begin{align}
\label{2}
    \text{E}_1&=\frac{1}{2\pi}\brak{\int_{-11\text{B}}^{-9\text{B}}\frac{1}{4}\ \text{df}+\int_{9\text{B}}^{11\text{B}}\frac{1}{4}\ \text{df}}=\frac{\text{B}}{2\pi}
\end{align}
So, from \brak{\ref{1}} and \brak{\ref{2}},
\begin{align}
    \text{E}_1&=\frac{\text{E}}{2}
\end{align}
Hence, option B is correct
\end{document}
